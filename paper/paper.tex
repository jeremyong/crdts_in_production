%%%%%%%%%%%%%%%%%%%%%%%%%%%%%%%%%%%%%%%%%
% Journal Article
% LaTeX Template
% Version 1.3 (9/9/13)
%
% This template has been downloaded from:
% http://www.LaTeXTemplates.com
%
% Original author:
% Frits Wenneker (http://www.howtotex.com)
%
% License:
% CC BY-NC-SA 3.0 (http://creativecommons.org/licenses/by-nc-sa/3.0/)
%
%%%%%%%%%%%%%%%%%%%%%%%%%%%%%%%%%%%%%%%%%

%----------------------------------------------------------------------------------------
%	PACKAGES AND OTHER DOCUMENT CONFIGURATIONS
%----------------------------------------------------------------------------------------

\documentclass[twoside]{article}

\usepackage{cite}

\usepackage{lipsum} % Package to generate dummy text throughout this template

\usepackage[sc]{mathpazo} % Use the Palatino font
\usepackage[T1]{fontenc} % Use 8-bit encoding that has 256 glyphs
\linespread{1.05} % Line spacing - Palatino needs more space between lines
\usepackage{microtype} % Slightly tweak font spacing for aesthetics

\usepackage[hmarginratio=1:1,top=32mm,columnsep=20pt]{geometry} % Document margins
\usepackage{multicol} % Used for the two-column layout of the document
\usepackage[hang, small,labelfont=bf,up,textfont=it,up]{caption} % Custom captions under/above floats in tables or figures
\usepackage{booktabs} % Horizontal rules in tables
\usepackage{float} % Required for tables and figures in the multi-column environment - they need to be placed in specific locations with the [H] (e.g. \begin{table}[H])
\usepackage{hyperref} % For hyperlinks in the PDF

\usepackage{lettrine} % The lettrine is the first enlarged letter at the beginning of the text
\usepackage{paralist} % Used for the compactitem environment which makes bullet points with less space between them

\usepackage{abstract} % Allows abstract customization
\renewcommand{\abstractnamefont}{\normalfont\bfseries} % Set the "Abstract" text to bold
\renewcommand{\abstracttextfont}{\normalfont\small\itshape} % Set the abstract itself to small italic text

\usepackage{titlesec} % Allows customization of titles
\renewcommand\thesection{\Roman{section}} % Roman numerals for the sections
\renewcommand\thesubsection{\Roman{subsection}} % Roman numerals for subsections
\titleformat{\section}[block]{\large\scshape\centering}{\thesection.}{1em}{} % Change the look of the section titles
\titleformat{\subsection}[block]{\large}{\thesubsection.}{1em}{} % Change the look of the section titles

\usepackage{fancyhdr} % Headers and footers
\pagestyle{fancy} % All pages have headers and footers
\fancyhead{} % Blank out the default header
\fancyfoot{} % Blank out the default footer
\fancyhead[C]{October 2013} % Custom header text
\fancyfoot[RO,LE]{\thepage} % Custom footer text

%----------------------------------------------------------------------------------------
%	TITLE SECTION
%----------------------------------------------------------------------------------------

\title{\vspace{-15mm}\fontsize{24pt}{10pt}\selectfont\textbf{CRDTs in Production}} % Article title

\author{
\large
\textsc{Jeremy Ong}\\[2mm] % Your name
\vspace{-5mm}
}
\date{}

%----------------------------------------------------------------------------------------

\begin{document}

\maketitle % Insert title

\thispagestyle{fancy} % All pages have headers and footers

%----------------------------------------------------------------------------------------
%	ABSTRACT
%----------------------------------------------------------------------------------------

\begin{abstract}

\noindent TODO

\end{abstract}

%----------------------------------------------------------------------------------------
%	ARTICLE CONTENTS
%----------------------------------------------------------------------------------------

\begin{multicols}{2} % Two-column layout throughout the main article text

\section{Introduction}

\lettrine[nindent=0em,lines=3]{T} here has been increasing interest in
the usage of CRDTs in environments of relaxed temporal
constraints. With the maturity of several distributed AP databases,
developers now have a choice to emphasize availability in exchange for
immediate consistency. These databases necessarily require distributed
data to be subject to possible logical inconsistencies due to presence
of multiple writers and readers in addition to replication
latency. CRDTs, commutative or convergent replicated data types, offer
a formal and structured approach to ensuring eventual consistency by
enforcing the following conditions: termination, eventual effect, and
convergence \cite{shapiro2011convergent}. Many CRDTs have been
identified of varying complexities and applicability (e.g. treedoc,
registers, pn-counters) \cite{letia2009crdts, shapiro2011convergent}.

However, adoption of these CRDTs into production databases has been
slow. In addition, it is not always practical to represent every
application-sensitive database object as a CRDT. While it is likely
that all objects can be represented as a set of disjoint CRDTs, this
can exacerbate the logical divergence of inter-key relationships due
to a lack of multi-key transactions.

A more pragmatic approach is to employ application level custom
convergence code, modeled after CRDT concepts to manage
conflict. Because of the generalized nature of this technique, it can
be difficult to formalize. This write-up intends to discuss various
common scenarios and show how by relaxing certain constraints,
behavior very close to a CRDT can be observed. In situations where
operations must be applied that cannot commute, mitigation strategies
are introduced that reduce the operation in question to a delayed
transaction.

%------------------------------------------------

\section{Preliminaries}

Let $x\in X$ denote an object in application space and
$f \in F:X \rightarrow X$ an operation that acts on an element of
$X$. Two versions of the same object are differentiated by subscript.
The causal history of an object $x$, $C(x)$ is defined as a set of
operations in terms of the following properties:
\begin{enumerate}
  \item Initially, the causal history is an empty set: $C(x_0) = \emptyset$
  \item A single atomic update augments the causal history: $C(x_{i+1}) =
    C(x_i)\cup \left\{f\right\}$
  \item The causal history of a merged object is the union of causal
    history of the objects merged: $C(x_i\cup x_j) = C(x_i)\cup C(x_j)$
\end{enumerate}
We can define a partial ordering on objects in $X$ based on causal
histories.
\[ x_i \leq x_j \Leftrightarrow C\left(x_i\right) \subset C\left(x_j\right) \]
In words, this means that all operations that have occurred to $x_i$
have also occurred to $x_j$.

CRDTs themselves are divided into two categories: CvRDTs (convergent
replicated data types) are state based while CmRDTs (commutative
replicated data types) are operation based. When clients request
an operation to be applied, they must first retrieve a replica on
which base the request (termed the source replica). The operation is
then applied either at the source (CvRDT) or downstream for all
replicas of that object (CmRDT). The at-source and downstream phases
are subject to preconditions. Both CvRDTs and CmRDTS must satisfy
the following conditions in the traditional definition of a CRDT
\cite{shapiro2011convergent}:

\begin{enumerate}
  \item Termination: The at-source and downstream phases terminate
    when preconditions are satisfied
  \item Eventual Effect:
    \[\forall i \exists j \ni x_j \leq x_k \Leftrightarrow f\in C\left(x_i\right) \Rightarrow f\in C\left(x_k\right) \]
    Colloquially, if an operation is performed on an object, that
    operation will eventually exist in the causal history of all replicas.
  \item Convergence:
    \[C\left(x_i\right) = C\left(x_j\right) \Rightarrow x_i \equiv x_j\]
    Replicas with identical histories contain the same state.
\end{enumerate}

%------------------------------------------------

\section{Managing Consistency in Practice}

This paper is restricted to circumstances in which the
developer has already made the tradeoff of leaning towards
availability and relaxing temporal consistency in selecting a
database. Otherwise, all operations would be wrapped in a transaction
for which immediate consistency is guaranteed. Given this tradeoff,
there are several alternatives that should also be considered before
immediately implementing any merge strategy.

First, the developer has an option to serialize all writes to a
particular object in a single queue. In practice, this will guarantee
immediate consistency again but at the cost of availability gained at
the database level. Nevertheless, this may be useful if immediate
consistency is only needed for a small fraction of the keys for which
availability is less important. If this strategy is to be used on the
majority of keys, the developer is better off choosing a CP database.
Second, the developer has the option to implement a locking
system. This is similar to the first technique but allows the locking
layer to be scaled independently of the writers which may be
distributed across many machines.

In general, because there is no easy way to manage multi-key
transactions in an eventually consistent setting, correlated data must
be denormalized in a single key value pair. For example, if an
operation to $x \in X$ is contingent on the result of an operation on
$y \in Y$, the developer is better off denormalizing $X$ and $Y$ into
a single object type $Z = X \times Y$ rather than attempting to
coordinate the logical dependency.

A merge strategy can be useful if the data in question exhibits low
merge intensity. Qualitatively, merge intensity corresponds to the
fraction of writes that will resolve in the formation of
replicas. Factors that correspond to low merge intensity include:
\begin{compactitem}
  \item Low write frequency to a single key for a given writer
  \item Few writers to a single key
  \item Short timespan between a key read and subsequent write
\end{compactitem}
Conversely, factors that correspond to high merge intensity include:
\begin{compactitem}
  \item Low write frequency to a single key for a given writer
  \item Few writers to a single key
  \item Short timespan between a key read and subsequent write
\end{compactitem}
By restricting the usage of a custom merge strategy to cases where
merge intensity is low, the developer avoids scenarios where merges
become inefficient or even impossible in certain scenarios.

Instead of a custom merge strategy of course, the developer may use
one of the canonical CRDTs. In practice, CRDTs are rarely implemented
at the database level, although database companies are beginning to
incorporate the simpler ones. At the time of this writing, Riak has
implemented a PN-counter for example \cite{riak2013} with more data
types on the way. Unfortunately, application data still exhibits more
complexity than can be handled natively with most databases, so we are
restricted with handling operations and merge logic on the application
side. In addition, not all application operations commute in general,
and yet these operations must still occur. Incorporating such
operations in the data model will require the relaxation on some of
the canonical CRDT constraints.

%------------------------------------------------

\section{Relaxing CRDT Constraints}



\section{Discussion}
\section{Conclusion}

%----------------------------------------------------------------------------------------
%	REFERENCE LIST
%----------------------------------------------------------------------------------------

\bibliography{paper}{}
\bibliographystyle{plain}

%----------------------------------------------------------------------------------------

\end{multicols}

\end{document}
